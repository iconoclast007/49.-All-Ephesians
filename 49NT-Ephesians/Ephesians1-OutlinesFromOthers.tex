\subsection{Outlines from Others}


\subsubsection{Unity in God's Sovereignty}
\textbf{Introduction:} The great theme of the letter of Paul to the church at Ephesus is unity. Unity is not the ecumenical ideology that we should all get together and forget our doctrinal differences or compromise with worldly standards and behavior. Rather, unity is based in the one and only plan of God for man—from creation to salvation to eternal glorification or separation. Man broke the unity through sin. Because of the nature and source of this sin, the former anointed cherub himself, the way back to God for man was not an easy one for God to make. It required the sacrifice of His only begotten Son. But through that sacrifice, God provided man a way for unity once again with HIM, and ultimately with one another as individuals come to Him for salvation. This chapter highlights the plan of God for unity with man.\footnote{09 May 2016, Clarence Billheimer}
\index[speaker]{Clarence Billheimer!Ephesians 1 (Unity in God's Sovereignty)}
\index[series]{Ephesians (Clarence Billheimer)!Ephesians 1 (Unity in God's Sovereignty)}
\index[date]{2016/05/09!Ephesians 1 (Unity in God's Sovereignty) (Clarence Billheimer)}
\begin{compactenum}[I.]
    \item \textbf{God’s sovereign purposes for the believer} \index[scripture]{Ephesians!Ephesians 01:01-06} (Ephesians 1:1-6)
    \begin{compactenum}[A.]
    	\item Heavenly blessings
    	\item Holy and without blame creatures
    	\item Heirs with Christ by adoption
    \end{compactenum}
    \item \textbf{God’s salvation for the believer} \index[scripture]{Ephesians!Ephesians 01:07-12} (Ephesians 1:7-12)
    \begin{compactenum}[A.]
    	\item Redemption and forgiveness
    	\item Revealing His will for us
    	\item Reunion of heaven and earth
    	\item Receiving an inheritance
    \end{compactenum}
    \item \textbf{God’s sealing of the believer} \index[scripture]{Ephesians!Ephesians 01:13-14} (Ephesians 1:13-14)
    \begin{compactenum}[A.]
    	\item The gospel of sealing
    	\item The guarantee of sealing
    \end{compactenum}
    \item \textbf{Paul’s supplications to the believer} \index[scripture]{Ephesians!Ephesians 01:15-18} (Ephesians 1:15-18)
    \begin{compactenum}[A.]
    	\item A daily supplication
    	\item A determined goal
    \end{compactenum}
    \item \textbf{God’s superiority in Christ} \index[scripture]{Ephesians!Ephesians 01:19-23} (Ephesians 1:19-23)
    \begin{compactenum}[A.]
    	\item The source of that power
    	\item The superness of that power
    \end{compactenum}
\end{compactenum}
\textbf{Conclusion:} The unity we should be seeking is a focus on the fact that as believers we are a body, a building, and a bride, and we are all being called to battle. The more Christians focus on these, the less division there will be. The sovereignty of God surpasses all human understanding. We cannot fathom how God can foreknow all and still give man choice between Him and the devil. It appears He has set everything up like a giant puppet show or robot activity. But because He is the only eternal entity in existence, He has the sole prerogative, authority, and right to choose how He performs His will. It is totally perfect in every way. That is what His sovereignty is all about, and we ought to be thankful it is that way because it gives us something and Someone we can totally rely on and trust in.
