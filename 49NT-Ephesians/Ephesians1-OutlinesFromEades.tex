\subsection{Outlines from Eades}



\subsubsection{God's Prized Possession}
\textbf{Introduction:} Ownership, to hold or occupancy, either with or without rights of ownership...property or wealth.
a territorial dominion of a something.  To physically control oneself or one's mind, domination, actuation, or obsession by a feeling.
Malachi 3:17 (KJV)  And they shall be mine, saith the LORD of hosts, in that day when I make up my jewels; and I will spare them, as a man spareth his own son that serveth him.
(Heres the church, heres the steeple, open the door and theres all the people!)  The church is more than a building on the side of the road with a steeple. When you look at the sun and it's power, all the stars going on forever in the heavens; none of that is to compare with God's love and cherished devotion to his church....his people.  
The church is referred to as a pearl of great price to God for which he has invested all he is period it has been said that God's interest (investment) in you will always be more than your interest in him!
*Here are 19 "P's" in a pod:\\
\noindent \textbf{Source:} \href{https://www.facebook.com/groups/sermonhints/posts/1874296819478790}{Facebook Sermon Hints Group, 1 April 2017}
\index[speaker]{Brian Eades!Ephesians 1 (God's Prized Possession)}
\index[series]{Ephesians (Brian Eades)!Ephesians 1 (God's Prized Possession)}
\index[date]{2017/04/01!Ephesians 1 (God's Prized Possession) (Brian Eades)}
\begin{compactenum}[I.][18]
    \item A PLANNED POSSESSION
    \item A PEACEFUL POSSESSION -2
    \item A PURCHASED POSSESSION
    \item A PAMPERED POSSESSION -3
    \item A PREDESTINED/PREHISTORIC POSSESSION-4,5,11
    \item A PLEASURABLE POSSESSION-5,9
    \item A PRAISEWORTHY (and praising) POSSESSION -6,12
    \item A PURCHASED POSSESSION -7,14
    \item A PRUDENT POSSESSION -8
    \item A PLEASURABLE POSSESSION -9
    \item A PURPOSED POSSESSION -11
    \item A PRESERVED POSSESSION -13
    \item A PRAYERFUL POSSESSION -16-18
    \item A POWERFUL POSSESSION-19
    \item A PLACED POSSESSION -20
    \item A PERCHED POSSESSION -21
    \item A PERSONAL POSSESSION -21-23
    \item A PERPETUAL POSSESSION!
\end{compactenum}


\subsubsection{The Chosen}
\textbf{Introduction:} Ephesians 1:4-5  According as he hath chosen us in him before the foundation of the world, that we should be holy and without blame before him in love:
Having predestinated us unto the adoption of children by Jesus Christ to himself, according to the good pleasure of his will,
Jesus told his disciples, "Ye have not chosen me, but I have chosen you"!  
For God to choose these humble disciples it was an undeserved choice, an unprecedented choice, an uncanny choice, a unique choice!
Sometimes preachers get all wrapped up in Calvinism when they come to this scripture. Instead of trying to dissect what it means to be "called", we should be celebrating the fact that God would even look our way to begin with!
For example, I have been called to preach the gospel, but not everyone has been called to preach the gospel. The gospel is for everyone and for whosoever will, but only a few people have the calling to stand and preach it. 
For the same purpose, God calls out people to establish churches and to preach and work in ministry so that the world may hear and be saved. That is the context by which Paul speaks of "predestination and the call or election" of the Saints. It is written in the context of ministry specifically as it relates to the church to reach "whosoever will".
2 Peter 3:9 (KJV)  The Lord is not slack concerning his promise, as some men count slackness; but is longsuffering to us-ward, not willing that any should perish, but that all should come to repentance.
It is important to remember that we cannot cut what Peter said out of the Bible under the inspiration of the Holy Ghost. The God that so loved the world that he gave his only begotten son, "is not willing that ANY SHOULD PERISH, but that ALL SHOULD COME TO REPENTANCE"!  God chose the Saints, the leaders of the ministry of the church to reach the world. That is why Jesus said many are called but few are chosen (i.e. the world has been delivered the message, but not everyone is a preacher or in Ministry)... as Matthew 22 declares, everyone is invited to the wedding but not everyone decides to put the wedding garment on by faith. The Chosen are those who accepted the gospel by faith and can be used to serve God in his kingdom. The called are those who have heard the message but have rejected it.  Simple.
As it relates to the Ministers of the Gospel:\\
\noindent \textbf{Source:} \href{https://www.facebook.com/groups/sermonhints/posts/2855730531335409}{Facebook Sermon Hints Group, 23 November 2020}
\index[speaker]{Brian Eades!Ephesians 1:4-5 (The Chosen)}
\index[series]{Ephesians (Brian Eades)!Ephesians 1:4-5 (The Chosen)}
\index[date]{2020/11/23!Ephesians 1:4-5 (The Chosen) (Brian Eades)}
\begin{compactenum}[I.]
    \item Chosen in the plan of his word
    \item Chosen before his presence without blame (v.4)
    \item Chosen for the purpose of witnessing (John 17)
    \item Chosen in his pleasurable will (v.5)
    \item Chosen with power to work
\end{compactenum}


\subsubsection{Through his blood}
\textbf{Introduction:} Ephesians 1:7  In whom we have redemption through his blood, the forgiveness of sins, according to the riches of his grace;
When Jesus began his ministry in Galilee he had 70 disciples. When you began to talk about eating his flesh and drinking his blood, his followers thinned out. The message of the blood still thins the crowds out because it relates to the price of sin and the suffering pay for sin. Basically in order to embrace the blood message of the Gospel you must first see yourself as a helpless sinner in need of such a price.
In the Old Testament concerning the last plague of Egypt being the death of the first born, God said "when I see the blood I will pass over you". He did not say when I see your church attendance or when I see your good works, he said that he must see the blood!
Hebrews chapter 9 tells us that without shedding of blood there is no remission. Simon Peter tells us in his epistle that we were not Redeemed by corruptible things as silver and gold, but with the precious blood of Christ as of a lamb without blemish and without spot!
When we partake of the Lord's Supper oh, the communion represents the broken body and the blood of Jesus Christ. When a person is baptized, their immersion under the water in a type, represents the death that came from loss of blood on the cross, followed by the powerful resurrection of Christ!
Moses wife called him a bloody husband because she circumcised her son at Moses command to be in obedience to the Lord at that time. She would not go to Egypt with him to deliver Israel, but instead went home to Jethro her father in Midian. It was only later that Jethro encouraged her to meet Moses at Mount Sinai and reunite.
The point is this, the message of the blood has always been a Battleground for the Saints! We are told to preach and proclaim the message of the power of the blood of Jesus Christ, and because of this we can expect conflict until Jesus comes again.\\
\noindent \textbf{Source:} \href{https://www.facebook.com/groups/sermonhints/posts/2856486564593139}{Facebook Sermon Hints Group, 24 November 2020}
\index[speaker]{Brian Eades!Ephesians 1:17 (Through his blood)}
\index[series]{Ephesians (Brian Eades)!Ephesians 1:17 (Through his blood)}
\index[date]{2020/11/24!Ephesians 1:17 (Through his blood) (Brian Eades)}
\begin{compactenum}[I.]
    \item Through the blood we saw The Passion of God... John 3:16
    \item Through the blood we have pardon
    \item Through the blood we have peace
    \item Through the blood we have power
    \item Through the blood we can preach
    \item Through the blood God the Father was pleased
    \item Through the blood we are made perfect
\end{compactenum}
\textbf{Extra:} \href{https://www.sharefaith.com/blog/2014/07/30-blood-of-jesus/?fbclid=IwAR0_1S4LzOqw-u03YpAxYvvTw066k3zHIgq32rq5hqQ0wzG1eYrJPQjXnl8}{30 different things that the blood of Jesus does for us}


\subsubsection{Far Above}
\textbf{Introduction:} Ephesians 1:13, 20-21  In whom ye also trusted, after that ye heard the word of truth, the gospel of your salvation: in whom also after that ye believed, ye were sealed with that holy Spirit of promise,
Which he wrought in Christ, when he raised him from the dead, and set him at his own right hand in the heavenly places, 
Far above all principality, and power, and might, and dominion, and every name that is named, not only in this world, but also in that which is to come:\\
\noindent \textbf{Source:} \href{https://www.facebook.com/groups/sermonhints/posts/2856629071245555}{Facebook Sermon Hints Group, 24 November 2020}
\index[speaker]{Brian Eades!Ephesians 1:13 (Far Above)}
\index[series]{Ephesians (Brian Eades)!Ephesians 1:13 (Far Above)}
\index[date]{2020/11/24!Ephesians 1:13 (Far Above) (Brian Eades)}
\begin{compactenum}[I.]
    \item Above the Pandemic 
    \item Above the Politicans
    \item Above the Predicaments
    \item Above the Powers of darkness 
    \item Above the Planets
    \item Above the Pains of this world
\end{compactenum}



\subsubsection{The Seal of the Spirit}
\textbf{Introduction:} Ephesians 1:13-14  In whom ye also trusted, after that ye heard the word of truth, the gospel of your salvation: in whom also after that ye believed, ye were sealed with that holy Spirit of promise,
Which is the earnest of our inheritance until the redemption of the purchased possession, unto the praise of his glory.
I remember back in 1981, i and my girlfriend (now wife) planted a garden together. That year we canned 300 jars of beans! When she was canning the beans she would pull them out of the boiler, and tell me to count how many times I heard the lids pop when they sealed. If the lid didn't pop it didn't seal.
In 1976 the Holy Spirit sealed me unto the day of redemption; and I know this because I was there when the "lid popped"! In the Bible and throughout human history, seals were very important in relation to protecting important documents and certifying the sender. As it relates to the Holy Spirit, the seal is a very important subject as it relates to the promise of God.\\
\noindent \textbf{Source:} \href{https://www.facebook.com/groups/sermonhints/posts/2856604397914689}{Facebook Sermon Hints Group, 24 November 2020}

\index[speaker]{Brian Eades!Ephesians 1:13-14 (The Seal of the Spirit)}
\index[series]{Ephesians (Brian Eades)!Ephesians 1:13-14 (The Seal of the Spirit)}
\index[date]{2020/11/24!Ephesians 1:13-14 (The Seal of the Spirit) (Brian Eades)}
\begin{compactenum}[I.]
    \item The Authority of the Seal
    \item The Agreement of the Seal (down payment- "ernest")
    \item The Acceptance of the Seal (note : "after ye heard, after ye believed")
    \item The Assurance of the Seal!
\end{compactenum}
