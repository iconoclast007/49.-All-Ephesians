\subsection{Outlines from Others}

\subsubsection{Walk Worthy of Your Vocation}
Ephesians 4:1:\footnote{Dr John Lineberry, \emph{The Biblical Evangelist}, Nov 2014-January 2015, Volume 45 Number -- \textcolor[rgb]{0.00,0.25,0.00}{\hyperlink{PsalmsTOC}{Return to end of Table of Contents.}}}
\begin{compactenum}[I.]
    \item \textbf{Walk and Fellowship} 
    \item \textbf{Walk and Fruitfulness}
    \item \textbf{Walk and Faithfulness} 
\end{compactenum}

\subsubsection{Unity of Servant Gifts}
\textbf{Introduction:} There are different kinds of spiritual gifts. Contrary to at least one group of Christianity’s teaching, spiritual gifts are not all about speaking in tongues and healing. They have focused mostly on them and practically, if not totally, ignored and forgotten about the rest. There are other spiritual gifts. Some were withdrawn eventually or refocused in other ways by the Holy Spirit when the Bible was completed, by the wa...y. There are service gifts. None of these have been withdrawn and no longer needed. And in this chapter we read of servant gifts. Two of them have been withdrawn—apostles and prophets—but not the rest. We need MORE of the rest! In each case, God’s desire is for them to fulfill the purpose of unity in Christ, not for boasting and exclusiveness they are so often emphasized with now.\footnote{11 May 2016, Clarence Billheimer}
\index[speaker]{Clarence Billheimer!Ephesians 4 (Unity of Servant Gifts)}
\index[series]{Ephesians (Clarence Billheimer)!Ephesians 4 (Unity of Servant Gifts)}
\index[date]{2016/05/11!Ephesians 4 (Unity of Servant Gifts) (Clarence Billheimer)}
\begin{compactenum}[I.]
    \item \textbf{The plan for servant gifts} \index[scripture]{Ephesians!Ephesians 04:01-03} (Ephesians 4:1-3) 
    \begin{compactenum}[A.]
    	\item For a worthy walk in our vocation
    	\item With lowliness, meekness, longsuffering, and love in mind
    	\item With an effort to keep our unity with Christ, and thus with one another
    \end{compactenum}
    \item \textbf{The person giving servant gifts} \index[scripture]{Ephesians!Ephesians 04:04-07} (Ephesians 4:4-7) 
    \begin{compactenum}[A.]
    	\item The body and Spirit are one
    	\item The Lord, faith, and baptism are one
    	\item The God and Father is above all, through all, and in all who are in His unity
    	\item The measure of grace given, though, may differ, depending on the individual need
    \end{compactenum}
    \item \textbf{The perogative for servant gifts} \index[scripture]{Ephesians!Ephesians 04:08-10} (Ephesians 4:8-10) 
    \begin{compactenum}[A.]
    	\item Ascending on high—this may refer to the statement Jesus made to Mary, “Touch Me not, for I have
not yet ascended unto My Father.”\index[scripture]{John!John 20:17} John 20:17
    	\item The rescue of Old Testament saints—this is what most Bible scholars believe this is referring to
    	\item The completion of the atonement—“fill all things” could refer to taking the blood to the mercy seat, fulfilling all things
    \end{compactenum}
    \item \textbf{The provision for servant gifts} \index[scripture]{Ephesians!Ephesians 04:11-16} (Ephesians 4:11-16) 
    \begin{compactenum}[A.]
    	\item The list implies a variation, not that one person has them all or that all persons should have all
    	\item The purpose is ongoing
    	\begin{compactenum}[1.]
    		\item Till the unity is finalized in heaven
    		\item To prevent wandering into every wind of doctrine
    		\item To mature as Christians
    		\item To perpetuate the unity of Christians
    	\end{compactenum}
    \end{compactenum}
    \item \textbf{The plan for servant gifts} \index[scripture]{Ephesians!Ephesians 04:17-32} (Ephesians 4:17-32) 
    \begin{compactenum}[A.]
    	\item The contrast explained
    	\item The changes in behavior expected
    \end{compactenum}
\end{compactenum}
\textbf{Conclusion:} It is evident from this chapter that God was developing a new and better way for His children to be taught the word of God and the behavior of the child of God, beyond that of law and ceremonies and sacrifices He had given through Moses. Now much of this, being pictures of Christ and the age of grace, has been fulfilled through Christ, and His grace is being expanded even more to the entire world—Jew and Gentile, as we noted in chapter 3.
