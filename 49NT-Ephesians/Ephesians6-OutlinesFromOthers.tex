\subsection{Outlines from Others}

\subsubsection{Unity of the Soldiers}
\textbf{Introduction:} Before we conclude our series on this book of the Bible, we’re going to do a further study in this chapter on the armor of God and wrap it up, then, with how unity can be lost among believers, as shown in the letter John wrote them. This message will focus on the unity of the soldiers. One of the stanzas of “Onward Christian Soldiers” says, “We are not divided; all one body, we; one in hope and doctrine, one in charity.” Sa...dly, I’m afraid that is not as true today as the songwriter would have liked and hoped for it to be. Satan has masterfully divided the unity of the soldiers with many new teachings that sound good, sound spiritual, and even sound like they come straight from the word of God. There are admonitions in this chapter that can help us get back to the unity in the Lord that He prayed for in His prayer (John 17).\footnote{17 May 2016, Clarence Billheimer}
\index[speaker]{Clarence Billheimer!Ephesians 6 (Unity of the Soldiers)}
\index[series]{Ephesians (Clarence Billheimer)!Ephesians 6 (Unity of the Soldiers)}
\index[date]{2016/05/17!Ephesians 6 (Unity of the Soldiers) (Clarence Billheimer)}
\begin{compactenum}[I.]
    \item \textbf{The chain of command is continued} the order of submission in the chain of command began in \index[scripture]{Ephesians!Ephesians 05:22} Ephesians 5:22 and continues through \index[scripture]{Ephesians!Ephesians 06:09} Ephesians 6:9. Each time, the one who is the submitter is mentioned first. Chapter divisions were not part of the originally inspired word of God, but it is interesting that the servants in the family (children and stewards) are separated by the chapter division. The heads of the family constitute one segment of the chain of command; those in the family comprise the balance.
    \item \textbf{The challenge in warfare is given} \index[scripture]{Ephesians!Ephesians 06:10--18} (Ephesians 6:10--18) 
    \begin{compactenum}[A.]
    	\item It is a challenge of courage and for courage.
    	\item It is a recognition of a powerful enemy and its forces.
    	\item It provides protection based on spiritual protection.
    	\item It gives not provision for retreat.
    	\item It has only one weapon, the sword of the Spirit, which is the word of God.
    	\item Its most important element is prayer.
    \end{compactenum}
    \item \textbf{The church will be instructed further} \index[scripture]{Ephesians!Ephesians 06:19--24} (Ephesians 6:19--24) 
    \begin{compactenum}[A.]
    	\item Paul’s personal prayer request is made.
    	\item Paul’s servant has been instructed as to what to tell the people of this church.
    \end{compactenum}
    \item \textbf{Submitting to the leaders through mutual submission} \index[scripture]{Ephesians!Ephesians 05:21-23} (Ephesians 5:21-23) these instructions carry over into \index[scripture]{Ephesians!Ephesians 06:01-09} Ephesians 6:1-9. The first chain of command dealt with is the family, God’s first institution. In each case, the one who is to be submitting is the one addressed first. In each case also, the one who has been given the position of authority is warned to not abuse this position but to be submissive in another way in return.
\end{compactenum}
\textbf{Conclusion:} aul’s final message is one typical of many of his letters, which usually start with, “Grace be to you and peace,” and often conclude the same way. It was Paul’s desire, in light of the tremendous resistance and persecution of the day, to encourage his converts that even in the midst of all this, they could still have and enjoy the real grace and peace of God.

