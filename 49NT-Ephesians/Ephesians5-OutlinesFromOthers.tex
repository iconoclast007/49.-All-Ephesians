\subsection{Outlines from Others}

\subsubsection{Thanksgiving}
Thanksgiving should be a major part of a Christan's life: \footnote{10 June 2015, Steve Angel, Lebanon Baptist Temple -- \textcolor[rgb]{0.00,0.25,0.00}{\hyperlink{EphesiansTOC}{Return to end of Table of Contents}}, \textcolor[rgb]{0.00,0.00,1.00}{with my addition(s)}}
\begin{compactenum}[I.]
	\item Give thanks \textbf{Always}  -- 
    \textcolor[rgb]{0.00,0.00,1.00}{\textbf{Expected}} (Eph 5:20, Psalm 34:1, Matthew 6:7,  Isa 63:7)
	\item Give thanks for \textbf{All Things} -- \textcolor[rgb]{0.00,0.00,1.00}{\textbf{Exhaustive}} (1Thess 5:18, Philippians 4:6)
	\item Give thanks to the \textbf{Almighty} -- \textcolor[rgb]{0.00,0.00,1.00}{\textbf{Exclusive}}
    \item \textcolor[rgb]{0.00,0.00,1.00}{Give Thanks \textbf{Audibly!}} -- \textcolor[rgb]{0.00,0.00,1.00}{\textbf{Expressive}}
    \item \textcolor[rgb]{0.00,0.00,1.00}{Give Thanks \textbf{Abundantly!}} -- \textcolor[rgb]{0.00,0.00,1.00}{\textbf{Excessive}}
    \item \textcolor[rgb]{0.00,0.00,1.00}{Give Thanks \textbf{Actually!}} -- 
    \textcolor[rgb]{0.00,0.00,1.00}{\textbf{Experiential}}
    \item \textcolor[rgb]{0.00,0.00,1.00}{Give Thanks in \textbf{Anticipation!}} -- 
    \textcolor[rgb]{0.00,0.00,1.00}{\textbf{Expectant}}
\end{compactenum}

\subsubsection{Unity in Submission}
\textbf{Introduction:} SUBMISSION! What a hated word! Man likes to be totally in charge of himself. He thinks that the sad conclusion of the book of Judges, “every man did that which was right in his own eyes,” is a good goal and position to have. God had at first given nature and animals in submission to man. He desired for that man to be in submission to Him. That all got upset and turned around because of sin. Man chose to submit to the devi...l because the devil made him think missing out on knowing about evil was bad for him. As a result, now nature fights against man with thorns and thistles, and the animal kingdom harbors “the fear of you and the dread of you” toward man, pronounced after the flood.
Chapter 5 of Ephesians continues with the instructions for perfecting of the saints which was in the concluding verses of chapter 4. There is a contrast here between what these believers were (compare with chapter 2:1-3) and what they should be striving to be now. Such goals, though, involve submission, and Paul’s desire is for each of them to have unity in submission. The goals he sets for them are just as applicable to us today.\footnote{14 May 2016, Clarence Billheimer}
\index[speaker]{Clarence Billheimer!Ephesians 5 (Unity in Submission)}
\index[series]{Ephesians (Clarence Billheimer)!Ephesians 5 (Unity in Submission)}
\index[date]{2016/05/14!Ephesians 5 (Unity in Submission) (Clarence Billheimer)}
\begin{compactenum}[I.]
    \item \textbf{Submitting to love versus lust} \index[scripture]{Ephesians!Ephesians 05:01-07} (Ephesians 5:1-7) the goal of Godly love versus the greed of selfish lust
    \begin{compactenum}[A.]
    	\item Christ is the example of love.
    	\item Sexual lusts and activity should not be named among believers; in other words, there should never be a time when this sin has to be dealt with.
    	\item Clean language and giving thanks should replace filthiness, foolish talk, and jesting.
    	\item People who continue the lifestyles here are not going to be in heaven.
    	\item The vain words of those who promote such lifestyles are deceiving; they face the wrath of God.
    	\item Believers should separate themselves from such men and activity.
    \end{compactenum}
    \item \textbf{Submitting to the Lord versus the works of darkness} \index[scripture]{Ephesians!Ephesians 05:08-14} (Ephesians 5:8-14) 
    \begin{compactenum}[A.]
    	\item Reprove them.
    	\item It is a shame to speak of them.
    \end{compactenum}
    \item \textbf{Submitting to the new life versus the old flesh} \index[scripture]{Ephesians!Ephesians 05:15-20} (Ephesians 5:15-20) 
    \begin{compactenum}[A.]
    	\item Wake up and rise up to the call of God.
    	\item Walk wisely in this world.
    	\item Use your time wisely.
    	\item Understand the Lord’s will.
    	\item Be continually filled with the Spirit.
    	\item Channel your mind with proper music.
    	\item Be continually thankful.
    \end{compactenum}
    \item \textbf{Submitting to the leaders through mutual submission} \index[scripture]{Ephesians!Ephesians 05:21-23} (Ephesians 5:21-23) these instructions carry over into \index[scripture]{Ephesians!Ephesians 06:01-09} Ephesians 6:1-9. The first chain of command dealt with is the family, God’s first institution. In each case, the one who is to be submitting is the one addressed first. In each case also, the one who has been given the position of authority is warned to not abuse this position but to be submissive in another way in return.
\end{compactenum}
\textbf{Conclusion:} The theme of this chapter is submission. Submission to God is the only way to receive His power and blessings.
