\subsection{Outlines from Billheimer}

\subsubsection{Unity of Two Sects -- Jews and Gentiles}
\textbf{Introduction:} The people of Paul’s day were mainly divided into three sects—Jews, Gentiles, and Samaritans. It took a little effort for Peter to realize the gospel was intended for all three. When Jesus told him he was getting the keys to the kingdom of heaven, that did not mean Peter was going to have the authority to stand at the gate of heaven and decide who could enter. That is a common thought woven into many religiou...s teachings and even many jokes. What Jesus meant was Peter would be given the privilege of introducing the grace of God through Christ to each group—and he did!
But, there was still work to do in developing and maturing the unity of the Jew and Gentile (here, Samaritans become the thought as a part of “Gentiles”). Discrimination had become as strong in this day as it was in our own country at one time. Paul becomes the next spokesman to reveal the “mystery” that God always intended the gospel for everyone; He had only chosen the Jews to be the vehicle it would be spread by, and they became arrogant and proud and against those who were not Jews to the point they were treated as outcasts. It was now time for this enmity to end.\footnote{11 May 2016, Clarence Billheimer}
\index[speaker]{Clarence Billheimer!Ephesians 3 (Unity of Two Sects -- Jews and Gentiles)}
\index[series]{Ephesians (Clarence Billheimer)!Ephesians 3 (Unity of Two Sects -- Jews and Gentiles)}
\index[date]{2016/05/11!Ephesians 3 (Unity of Two Sects -- Jews and Gentiles) (Clarence Billheimer)}
\begin{compactenum}[I.]
    \item \textbf{The mystery of the unity} \index[scripture]{Ephesians!Ephesians 03:01-06} (Ephesians 3:1-6) it was not totally revealed to the prophets and saints of the Old Testament era, yet God DID extend grace to Gentiles many times, and they could become believers through faith, too. In the age of grace, it was now going to be possible through Christ’s death and resurrection.
    \item \textbf{The message of the gospel} \index[scripture]{Ephesians!Ephesians 03:07-12} (Ephesians 3:7-12) Paul states that he was made the minister of this message. Indeed, he took it to many Gentile dominions in three long, arduous journeys, which we read about in nearly half the book of Acts.
    \item \textbf{The motives for the Ephesians} \index[scripture]{Ephesians!Ephesians 03:13-19} (Ephesians 3:13-19)
    \begin{compactenum}[A.]
    	\item That they not be discouraged because of Paul’s tribulations
    	\item That they would be strengthened by the Holy Spirit
    	\item That Christ would dwell in their hearts by faith
    	\item That they would be rooted and grounded in love
    	\item That they would be able to comprehend this wonderful new truth God has finally fully revealed
    \end{compactenum}
\end{compactenum}
\textbf{Conclusion:} The final two verses of this chapter are often used as an encouragement to us about our Savior, who is able to do exceeding abundantly above all we can ask or think. That is an excellent application. In context here, though, the thought is how exceeding abundantly above all we could ask or think it is when it comes to unifying the Gentile and the Jew, when for centuries there had been so much animosity and exclusive attitudes displayed by the Jew toward the Gentile.

